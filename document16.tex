\documentclass[journal,12pt,twocolumn]{IEEEtran}
%
\usepackage{setspace}
\usepackage{gensymb}
%\doublespacing
\singlespacing

%\usepackage{graphicx}
%\usepackage{amssymb}
%\usepackage{relsize}
\usepackage[cmex10]{amsmath}
%\usepackage{amsthm}
%\interdisplaylinepenalty=2500
%\savesymbol{iint}
%\usepackage{txfonts}
%\restoresymbol{TXF}{iint}
%\usepackage{wasysym}
\usepackage{amsthm}
%\usepackage{iithtlc}
\usepackage{mathrsfs}
\usepackage{txfonts}
\usepackage{stfloats}
\usepackage{bm}
\usepackage{cite}
\usepackage{cases}
\usepackage{subfig}
%\usepackage{xtab}
\usepackage{longtable}
\usepackage{multirow}
%\usepackage{algorithm}
%\usepackage{algpseudocode}
\usepackage{enumitem}
\usepackage{mathtools}
\usepackage{steinmetz}
\usepackage{tikz}
\usepackage{circuitikz}
\usepackage{verbatim}
\usepackage{tfrupee}
\usepackage[breaklinks=true]{hyperref}
%\usepackage{stmaryrd}
\usepackage{tkz-euclide} % loads  TikZ and tkz-base
%\usetkzobj{all}
\usetikzlibrary{calc,math}
\usepackage{listings}
    \usepackage{color}                                            %%
    \usepackage{array}                                            %%
    \usepackage{longtable}                                        %%
    \usepackage{calc}                                             %%
    \usepackage{multirow}                                         %%
    \usepackage{hhline}                                           %%
    \usepackage{ifthen}                                           %%
  %optionally (for landscape tables embedded in another document): %%
    \usepackage{lscape}     
\usepackage{multicol}
\usepackage{chngcntr}
%\usepackage{enumerate}

%\usepackage{wasysym}
%\newcounter{MYtempeqncnt}
\DeclareMathOperator*{\Res}{Res}
%\renewcommand{\baselinestretch}{2}
\renewcommand\thesection{\arabic{section}}
\renewcommand\thesubsection{\thesection.\arabic{subsection}}
\renewcommand\thesubsubsection{\thesubsection.\arabic{subsubsection}}

\renewcommand\thesectiondis{\arabic{section}}
\renewcommand\thesubsectiondis{\thesectiondis.\arabic{subsection}}
\renewcommand\thesubsubsectiondis{\thesubsectiondis.\arabic{subsubsection}}

% correct bad hyphenation here
\hyphenation{op-tical net-works semi-conduc-tor}
\def\inputGnumericTable{}                                 %%

\lstset{
%language=C,
frame=single, 
breaklines=true,
columns=fullflexible
}
%\lstset{
%language=tex,
%frame=single, 
%breaklines=true
%}

\begin{document}
%


\newtheorem{theorem}{Theorem}[section]
\newtheorem{problem}{Problem}
\newtheorem{proposition}{Proposition}[section]
\newtheorem{lemma}{Lemma}[section]
\newtheorem{corollary}[theorem]{Corollary}
\newtheorem{example}{Example}[section]
\newtheorem{definition}[problem]{Definition}
%\newtheorem{thm}{Theorem}[section] 
%\newtheorem{defn}[thm]{Definition}
%\newtheorem{algorithm}{Algorithm}[section]
%\newtheorem{cor}{Corollary}
\newcommand{\BEQA}{\begin{eqnarray}}
\newcommand{\EEQA}{\end{eqnarray}}
\newcommand{\define}{\stackrel{\triangle}{=}}

\bibliographystyle{IEEEtran}
%\bibliographystyle{ieeetr}


\providecommand{\mbf}{\mathbf}
\providecommand{\pr}[1]{\ensuremath{\Pr\left(#1\right)}}
\providecommand{\qfunc}[1]{\ensuremath{Q\left(#1\right)}}
\providecommand{\sbrak}[1]{\ensuremath{{}\left[#1\right]}}
\providecommand{\lsbrak}[1]{\ensuremath{{}\left[#1\right.}}
\providecommand{\rsbrak}[1]{\ensuremath{{}\left.#1\right]}}
\providecommand{\brak}[1]{\ensuremath{\left(#1\right)}}
\providecommand{\lbrak}[1]{\ensuremath{\left(#1\right.}}
\providecommand{\rbrak}[1]{\ensuremath{\left.#1\right)}}
\providecommand{\cbrak}[1]{\ensuremath{\left\{#1\right\}}}
\providecommand{\lcbrak}[1]{\ensuremath{\left\{#1\right.}}
\providecommand{\rcbrak}[1]{\ensuremath{\left.#1\right\}}}
\theoremstyle{remark}
\newtheorem{rem}{Remark}
\newcommand{\sgn}{\mathop{\mathrm{sgn}}}
\providecommand{\abs}[1]{\left\vert#1\right\vert}
\providecommand{\res}[1]{\Res\displaylimits_{#1}} 
\providecommand{\norm}[1]{\left\lVert#1\right\rVert}
%\providecommand{\norm}[1]{\lVert#1\rVert}
\providecommand{\mtx}[1]{\mathbf{#1}}
\providecommand{\mean}[1]{E\left[ #1 \right]}
\providecommand{\fourier}{\overset{\mathcal{F}}{ \rightleftharpoons}}
%\providecommand{\hilbert}{\overset{\mathcal{H}}{ \rightleftharpoons}}
\providecommand{\system}{\overset{\mathcal{H}}{ \longleftrightarrow}}
	%\newcommand{\solution}[2]{\textbf{Solution:}{#1}}
\newcommand{\solution}{\noindent \textbf{Solution: }}
\newcommand{\cosec}{\,\text{cosec}\,}
\providecommand{\dec}[2]{\ensuremath{\overset{#1}{\underset{#2}{\gtrless}}}}
\newcommand{\myvec}[1]{\ensuremath{\begin{pmatrix}#1\end{pmatrix}}}
\newcommand{\mydet}[1]{\ensuremath{\begin{vmatrix}#1\end{vmatrix}}}
%\numberwithin{equation}{section}
\numberwithin{equation}{subsection}
%\numberwithin{problem}{section}
%\numberwithin{definition}{section}
\makeatletter
\@addtoreset{figure}{problem}
\makeatother

\let\StandardTheFigure\thefigure
\let\vec\mathbf
%\renewcommand{\thefigure}{\theproblem.\arabic{figure}}
\renewcommand{\thefigure}{\theproblem}
%\setlist[enumerate,1]{before=\renewcommand\theequation{\theenumi.\arabic{equation}}
%\counterwithin{equation}{enumi}


%\renewcommand{\theequation}{\arabic{subsection}.\arabic{equation}}

\def\putbox#1#2#3{\makebox[0in][l]{\makebox[#1][l]{}\raisebox{\baselineskip}[0in][0in]{\raisebox{#2}[0in][0in]{#3}}}}
     \def\rightbox#1{\makebox[0in][r]{#1}}
     \def\centbox#1{\makebox[0in]{#1}}
     \def\topbox#1{\raisebox{-\baselineskip}[0in][0in]{#1}}
     \def\midbox#1{\raisebox{-0.5\baselineskip}[0in][0in]{#1}}

\vspace{3cm}


\title{Assignment 12}
\author{Jayati Dutta}





% make the title area
\maketitle

\newpage

%\tableofcontents

\bigskip

\renewcommand{\thefigure}{\theenumi}
\renewcommand{\thetable}{\theenumi}
%\renewcommand{\theequation}{\theenumi}


\begin{abstract}
This is a simple document explaining how to form a basis of a vector space and how to get the coordinates of the vector.
\end{abstract}

%Download all python codes 
%
%\begin{lstlisting}
%svn co https://github.com/JayatiD93/trunk/My_solution_design/codes
%\end{lstlisting}

Download all and latex-tikz codes from 
%
\begin{lstlisting}
svn co https://github.com/gadepall/school/trunk/ncert/geometry/figs
\end{lstlisting}
%


\section{Problem}
In $C^3$, let $\vec{\alpha_1}$ = $(1,0,-i)$, $\vec{\alpha_2}$ = $(1+i,1-i,1)$ , $\vec{\alpha_3}$ = $(i,i,i)$. Prove that these vectors form a basis for $C^3$. What are the coordinates of the vector (a,b,c) in the basis?
\section{Explanation}
Now,
\begin{align}
C_1 \vec{\alpha_1} +C_2 \vec{\alpha_2}+C_3 \vec{\alpha_3} = \vec{0}\\
\implies C_1 \myvec{1\\0\\-i}+ C_2\myvec{1+i\\1-i\\1} + C_3\myvec{i\\i\\i} = \vec{0}
\end{align}
So,
\begin{align}
\myvec{1 & 1+i & i\\0 & 1-i & i\\-i & 1 & i}\myvec{C_1\\C_2\\C_3} = \myvec{0\\0\\0}
\end{align}
Considering the co-efficient matrix $A$:
\begin{multline}
\myvec{1 & 1+i & i\\0 & 1-i & i\\-i & 1 & i}\xleftrightarrow[]{R_3\leftarrow R_3+iR_1}\myvec{1 & 1+i & i\\0 & 1-i & i\\0 & i & i-1}\\
\xleftrightarrow[]{R_3\leftarrow R_3/i}\myvec{1 & 1+i & i\\0 & 1-i & i\\0 & 1 & 1+i}\xleftrightarrow[]{R_3\leftarrow (1-i)R_3}\\
\myvec{1 & 1+i & i\\0 & 1-i & i\\0 & 1-i & 2}\xleftrightarrow[]{R_3\leftarrow R_3-R_2}\myvec{1 & 1+i & i\\0 & 1-i & i\\0 & 0 & 2-i}\\
\xleftrightarrow[]{R_2\leftarrow \frac{1+i}{1-i}R_2}\myvec{1 & 1+i & i\\0 & 1+i & -1\\0 & 0 & 2-i}\xleftrightarrow[]{R_1\leftarrow R_1-R_2}\\
\myvec{1 & 0 & i+1\\0 & 1+i & -1\\0 & 0 & 2-i}
\end{multline}
Now let 
\begin{align}
R = \myvec{1 & 0 & i+1\\0 & 1+i & -1\\0 & 0 & 2-i}
\end{align}
Where $R$ is the row reduced form of matrix $A$. So $\vec{\alpha_1}$,$\vec{\alpha_2}$ and $\vec{\alpha_3}$ are linearly independent which implies that these 3 vectors form a basis of vector space $C^3$.

Now, consider a vector $\beta$ = $\myvec{a\\b\\c}$ and let the coordinates are $x=\myvec{x_1\\x_2\\x_3}$ such that
\begin{align}
Ax = \myvec{a\\b\\c}\\
\implies x = A^{-1} \myvec{a\\b\\c}
\end{align}
Let us consider a matrix $(A | I)$ where I is a 3x3 identity matrix.
Now, applying the Gauss-Jordon theorem we can get $A^{-1}$
\begin{multline}
\myvec{1 & 1+i & i & 1 & 0 & 0\\0 & 1-i & i & 0 & 1 & 0\\-i & 1 & i & 0 & 0 & 1}\\\xleftrightarrow[]{R_3\leftarrow R_3+iR_1}\\
\myvec{1 & 1+i & i & 1 & 0 & 0\\0 & 1-i & i & 0 & 1 & 0\\0 & i & i-1 & i & 0 & 1}\\
\xleftrightarrow[]{R_3\leftarrow R_3/i}\\
\myvec{1 & 1+i & i & 1 & 0 & 0\\0 & 1-i & i & 0 & 1 & 0\\0 & 1 & 1+i & 1 & 0 & -i}\\
\xleftrightarrow[]{R_3\leftarrow (1-i)R_3}\\
\myvec{1 & 1+i & i &1 & 0 & 0\\0 & 1-i & i & 0 & 1 & 0\\0 & 1-i & 2 & 1-i & 0 & -i-1}\xleftrightarrow[]{R_3\leftarrow R_3-R_2}\\
\myvec{1 & 1+i & i&1 & 0 & 0\\0 & 1-i & i&0 & 1 & 0\\0 & 0 & 2-i & 1-i & -1 & -i-1}\\
\xleftrightarrow[]{R_2\leftarrow \frac{1+i}{1-i}R_2}\\
\myvec{1 & 1+i & i & 1 & 0 & 0\\0 & 1+i & -1 & 0 & i & 0\\0 & 0 & 2-i & 1-i & -1 & -i-1}\\
\xleftrightarrow[]{R_1\leftarrow R_1-R_2}\\
\myvec{1 & 0 & i+1 & 1 & -i & 0\\0 & 1+i & -1 & 0 & i & 0\\0 & 0 & 2-i & 1-i & -1 & -i-1}\\
\xleftrightarrow[R_1\leftarrow R_1+R_3]{R_3\leftarrow\frac{R_3}{2-i}}\\
\myvec{1 & 0 & 3 & 2-i & -i-1 & -i-1\\0 & 1+i & -1 & 0 & i & 0\\0 & 0 & 1 & \frac{3-i}{5} & -\frac{2+i}{5} & -\frac{3i+1}{5}}\\
\xleftrightarrow[]{R_2\leftarrow R_2+R_3}\\
\myvec{1 & 0 & 3 & 2-i & -i-1 & -i-1\\0 & 1+i & 0 & \frac{3-i}{5} & \frac{-2+4i}{5} & -\frac{3i+1}{5}\\0 & 0 & 1 & \frac{3-i}{5} & -\frac{2+i}{5} & -\frac{3i+1}{5}}\\
\xleftrightarrow[]{R_2\leftarrow \frac{R_2}{1+i}}\\
\myvec{1 & 0 & 3 & 2-i & -i-1 & -i-1\\0 & 1 & 0 & \frac{3-i}{5(1+i)} & \frac{-2+4i}{5(1+i)} & -\frac{3i+1}{5(1+i)}\\0 & 0 & 1 & \frac{3-i}{5} & -\frac{2+i}{5} & -\frac{3i+1}{5}}\\
\xleftrightarrow[]{R_1\leftarrow R_1-3R_3}\\
\end{multline}
\begin{align}
\myvec{1 & 0 & 0 & \frac{1-4i}{5} & \frac{1-2i}{5} & \frac{-2+4i}{5}\\0 & 1 & 0 & \frac{1-2i}{5} & \frac{1+3i}{5} & -\frac{2+i}{5}\\0 & 0 & 1 & \frac{3-i}{5} & -\frac{2+i}{5} & -\frac{3i+1}{5}}\\
\implies (I|A^{-1}) = \myvec{1 & 0 & 0 & \frac{1-4i}{5} & \frac{1-2i}{5} & \frac{-2+4i}{5}\\0 & 1 & 0 & \frac{1-2i}{5} & \frac{1+3i}{5} & -\frac{2+i}{5}\\0 & 0 & 1 & \frac{3-i}{5} & -\frac{2+i}{5} & -\frac{3i+1}{5}}\\
\implies A^{-1}=\myvec{\frac{1-4i}{5} & \frac{1-2i}{5} & \frac{-2+4i}{5}\\\frac{1-2i}{5} & \frac{1+3i}{5} & -\frac{2+i}{5}\\\frac{3-i}{5} & -\frac{2+i}{5} & -\frac{3i+1}{5}}
\end{align}
 So,
\begin{align}
x = A^{-1} \myvec{a\\b\\c}\\
\implies x = \myvec{\frac{1-4i}{5} & \frac{1-2i}{5} & \frac{-2+4i}{5}\\\frac{1-2i}{5} & \frac{1+3i}{5} & -\frac{2+i}{5}\\\frac{3-i}{5} & -\frac{2+i}{5} & -\frac{3i+1}{5}}\myvec{a\\b\\c}\\
\implies \myvec{x_1\\x_2\\x_3} =  \myvec{\frac{1-4i}{5} & \frac{1-2i}{5} & \frac{-2+4i}{5}\\\frac{1-2i}{5} & \frac{1+3i}{5} & -\frac{2+i}{5}\\\frac{3-i}{5} & -\frac{2+i}{5} & -\frac{3i+1}{5}}\myvec{a\\b\\c}
\end{align}

%\renewcommand{\theequation}{\theenumi}
%\begin{enumerate}[label=\thesection.\arabic*.,ref=\thesection.\theenumi]
%\numberwithin{equation}{enumi}
%\item Verification of the above problem using python code.\\
%\solution The  following Python code verifies the above solution.
%\begin{lstlisting}
%codes/multiplication_test.py
%\end{lstlisting}
%%%
%\end{enumerate}

\end{document}



